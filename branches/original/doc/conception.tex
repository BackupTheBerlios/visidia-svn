
\section{Conception}

\subsection{Objectifs}
Le but de cet conception et de trouver un moyen de coder un cycle, pour pouvoir  implementer des algorithmes de type Dijkstra, ce type d'algorithmes ont souvent pour but de detecter la terminaison de l'algo, et ce en se basant sur la circulation d'un jeton le long d'un cycle par exemple .

 L'objectif est donc, de permettre � chaque sommet de conna�tre son prochain dans le cycle, afin de lui adresser des messages particuliers . Un sommet doit �galement pouvoir reconnaitre son pr�c�dent, afin de recevoir, o� d'attendre des messages qui y proviennent.

\subsection{Choix techniques}

Le fait de pouvoir situer son voisin est l'affaire de tout les sommets du graphe, il est donc logique de penser � modifier la structure des sommets afin de stocker leurs suivants.
 \\
Dans l'ancienne version de visidia, les objets sommets (SimpleGraphVertex.java) avaient comme variables d'instances, l'identificateur du sommet lui m�me, vecteurs de voisins, vecteurs d'ar�tes, l'�tat du noeud, et l'instance de la classe Data. Nous avons d�cider d'ajouter deux variables d'instance pour caract�riser le sommet suivant et le sommet pr�c�dent du sommet �tudi�, {\it Integer nextDoor} et {\it Integer previousDoor}, dans un �ventuel cycle. Ainsi que deux accesseurs {\it getNext()} et {\it getPrevious} qui retournent respectivement le suivant le pr�c�dent du sommets, et deux modificateurs {\it setNext(Integer next)} et {\it setPrevious(Integer previous)} qui positionnent les champs {\it nextDoor} et {\it previousDoor}. \\

 Le probl�me de la structure des donn�es est ainsi r�solu, reste � voir comment adapter la partie {\simulation

 
