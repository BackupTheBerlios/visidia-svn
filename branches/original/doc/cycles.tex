\section{Saisie du Cycle}
 \subsection{Comment orienter une ar�te?}

   Des algorithmes de type Diskjtra, pour les simuler on a besoin de tracer un cycle dans le graphe, et d'orienter ce cyle. L'outil ViSiDia permet � l'utilisateur de tracer des ar�tes orient�s. \\
  Les ar�tes sont par d�faut non orient�s. Il est touefois possible de tracer des ar�tes orient�s. \\
  Dans le menu Transformation, aller dans Change Edge Shape, une boite de dialogue apparait, � ce moment l�, choisir le type d'ar�te qu 'on veut tracer et appuyer sur OK. \\
  Il est � signaler qu'on peut changer le type d'ar�te � tout moment durant le trac� du graphe. \\

 \subsection{Comment saisir un cycle}
  
 Si l'algorithme le n�cessite, le mieux pour tracer un graphe avec cyle, et de commencer par tracer le cycle. i.e lier tout les sommets par avec des ar�tes orient�es d'abord, ensuite ajouter les ar�tes non orient�es s'il en faut.
 
 Il faut noter que derri�re l'affichage, 
 Ceci dit, rien n'oblige � suivre cette r�gle.
 \subsection{Remarque importante}

  Si plusieurs arr�tes orient�es sortent d'un m�me sommet, alors le suivant � ce sommet, est le sommet sur lequel arrive la derni�re arr�te orient�e parmi ces arr�tes.