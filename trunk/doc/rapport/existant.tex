Apr�s avoir  pr�sent� le  domaine de l'algorithmique  distribu�e, nous
allons  maintenant  faire  un  tour  d'horizon  de  l'implantation  de
\visidia telle qu'elle existait avant notre apport. 

\section{But de \visidia}

\section{Architecture g�n�rale}
\subsection{Interface graphique d'�dition}
\subsection{Interface graphique de simulation}
\subsection{Simulateur}
\subsection{Algorithmes}

\section{Deux modes d'utilisations}
\subsection{Syst�me unique}
\subsection{Distribu�}

\section{Mod�le implant�}
\subsection{Communication par messages asynchrones}
\subsection{Description de l'API existante}
\subsection{Exemple d'utilisation}

\section{Limite du mod�le actuel}

\section{Un nouveau mod�le : les Agents}

%% Local Variables: 
%% mode: latex
%% TeX-master: t
%% TeX-PDF-mode: t
%% coding: latin-1
%% End: