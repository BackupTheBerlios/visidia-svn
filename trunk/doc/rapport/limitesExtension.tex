\section{Les limites du projet}

Au cours de ce projet, la principale limite � laquelle nous avons �t� confront�s
est l'absence de documentation technique pr�cise et compl�te sur l'application.\\

Il est d'autant plus important de souligner ce probl�me auquel nous avons �t�
confront�s, car si le projet continue ainsi, les prochaines �quipes passeront
surement encore plus de temps � comprendre ce qui a �t� fait et le code produit
deviendra non maintenable du au fait d'une compr�hension partielle de certains
aspects de \visidia.\\

Il est � notre sens, imp�ratif, qu'un manuel technique sur tout l'application
soit r�dig�e et entretenu par chaque �quipe de PFA.



\section{Extension du projet}

Une fonctionnalit� qu'il serait utile de rajouter � \visidia serait de faire une
totale s�paration, entre l'algorithme ex�cut� par l'agent, et son algorithme de
d�placement.\\

Une autre �volution envisageable consisterait � d�velopper un
applet java afin de pouvoir utiliser 
l'application � l'aide d'un simple navigateur internet.\\

Il pourrait �tre int�ressant �galement de r�aliser de la simulation
d'algorithmes distribu�s sur de vrais r�seaux de machines physiques. \\
