%%% Extensibilit�
\section{Les Agents sur les architectures distribu�es}
Une tr�s grande issue sur la quelle le logiciel peut �voluer touche le domaine des architectures 
distribu�es vu que aujourd'hui l'algorithmique distribu�e est devenu un outil tr�s puissant dans la r�solution des probl�mes relatives � ce genre d'architectures.
\visidia Agent offre des outils de base pour un developpement future qui met en oeuvre des solutions pour ce genre de structure. L'utilisation des Agents mobiles sur des structures pareil va �conomiser enormement de resources. Le nombre de processus qui tournent ne sera plus compar�  aux nombre des noeuds du syst�mes comme se qui se passe avec la version normale de \visidia (sauf si l'utilisateur le voudra, en placent autant d'agent qu'il existe de noeud du systeme). Et dans d'autre cas il permettera de rentabiliser les resources pour une vitesse d'execution beaucoup plus grande (c'est le cas qu'on place un nombre d'agent sup�rieur � la cardinalit� du r�seau).
Avec \visidia en version d'agent mobile, le probl�me qui reste pos�e est celui de la visualisation graphique, en fait une telle chose diminue beaucoup la vitesse de l'execution vu qu'elle consomme plus de resources du systeme. Sur une telle structure un autre probl�me peut apparaitre c'est celui de l'optimisation des deplacements des Agents (une des solutions que offre la version actuelle est l'utilisation d'un AgentMover dans ce but). Le developpement d'un systeme complet de rencontre peut aussi contribu�e dans la procedure de l'optimisation des resources et d'augmentation de la rapidit� du systeme.

\section{Les Gros graphes}
� part le probleme de la visualisation graphique d�j� vu dans le paragraphe pr�cedent avec les architectures distribu�es, L'utilisation des agents dans un grand graphe exige certaine evolution du systeme.
Dans la version actuelle de \visidia Agent, les graphes ne sont personnalis�s que de mani�re graphique (via l'interface graphique). Le choix du type du d�placement des agents et le placement des agents se font aussi au niveau de cette interface. Le choix de faire une repr�sentation graphique des graphe ayant une grande cardinalit� est presque une aventure. L'API actuelle offre des bases pour developper un module qui permet le lancement de l'ex�cution en arri�re plan (sans visualisation), il reste aussi � faire un liens entre ce module et le module des stastiques d�j� pr�sent pour ne visualiser �  la fin que les statistiques de l'ex�cutions toute en gardant une trace.

\section{R�gles de r�ecritures}

%% Local Variables: 
%% mode: latex
%% TeX-master: "rapport"
%% TeX-PDF-mode: t
%% coding: latin-1
%% End: