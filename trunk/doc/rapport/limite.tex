
\paragraph{Lisibilit� et extensibilit�}

Cette  partie ne concerne  pas directement  la simulation  des agents,
mais \visidia dans  son ensemble : le programme souffre  de son �ge et
du fait d'�tre pass� entre de nombreuses mains d'�l�ves ing�nieurs peu
exp�riment�s (y compris nous).

De ce fait, il r�sulte  un programme qui fonctionne correctement, mais
aux possibilit�s  d'extension faibles (peu  d'utilisation du m�canisme
d'h�ritage). Ainsi,  les rajouts se greffent directement  dans le code
existant,  ou  via  des   copier-coller,  augmentant  les  risques  de
plantages et/ou d'incoh�rences.

Les parties sur  lesquelles nous avons travaill� accusant  ce genre de
probl�me sont  l'interface graphique principalement,  et le simulateur
dans une moindre mesure.

\paragraph{Consommation des ressources}





%%% Local Variables: 
%%% mode: latex 
%%% TeX-master:  "rapport" 
%%% TeX-PDF-mode: t 
%%% coding:latin-1 
%%% End:
