
\section{Lisibilit� et extensibilit�}

Cette  partie ne concerne  pas directement  la simulation  des agents,
mais \visidia dans  son ensemble : le programme souffre  de son �ge et
du fait d'�tre pass� entre de nombreuses mains d'�l�ves ing�nieurs peu
exp�riment�s (y compris nous).\\

De ce fait, il r�sulte  un programme qui fonctionne correctement, mais
aux possibilit�s  d'extension faibles (peu  d'utilisation du m�canisme
d'h�ritage). Ainsi,  les rajouts se greffent directement  dans le code
existant,  ou  via  des   copier-coller,  augmentant  les  risques  de
plantages et/ou d'incoh�rences.\\

Les parties sur  lesquelles nous avons travaill� accusant  ce genre de
probl�me sont  l'interface graphique principalement,  et le simulateur
dans une moindre mesure.

Id�alement,  l'interface  graphique  devrait  permettre le  rajout  de
module sans  avoir � modifier le code  d'origine. Malheureusement pour
cela, une refonte du  code de l'interface graphique semble in�vitable,
or il  s'agit d'un processus  lourd en temps  et personnes, ce  qui le
rend peu envisageable.

Concernant le simulateur, une factorisation des diff�rents simulateurs
existants  seraient   pr�f�rables.   Cette  factorisation   n'est  pas
essentielle actuellement mais le  deviendra si d'autres simulateurs se
greffent plus tard aux existants.

\section{Simulation sur des graphes de grandes tailles}

Permettre  une visualisation  graphique est  un avantage  certain pour
mieux comprendre et v�rifier le comportement d'un algorithme. Mais une
fois  que le  nombre de  noeuds d'un  graphe d�passe  la  centaine, il
devient  difficile  de  suivre  correctement  le  d�roulement  de  cet
algorithme.\\

Il  s'agit  donc  d'une   limitation  importante,  les  chercheurs  ne
souhaitant  pas  limiter leurs  exp�rimentations  sur  des graphes  de
petite  taille. La  solution  � adopter  pourrait  �tre d'utiliser  le
simulateur  sans passer par  l'interface graphique  puis d'enregistrer
chaque   action  dans  un   fichier,  mais   on  perd   clairement  en
visualisation.

\section{Choix du nombre d'agent}

Utiliser des  agents pour simuler des  algorithmes distribu�s pr�sente
l'avantage  de  pouvoir contr�ler  le  nombre  de  processus qui  sont
ex�cut�s  lors de  la simulation  (contrairement �  la  simulation par
messages qui impose un processus par sommet du graphe).
Le nombre  de processus d�pend  en effet du  nombre d'agent et  non du
nombre de  sommet. \\

Cette  avantage  a aussi  son  inconv�nient :  ce  que  l'on gagne  en
ressource  machine,  on  le  perd   bien  souvent  en  temps  lors  de
l'ex�cution d'algorithmes distribu�s avec un faible nombre d'agent. Si
au contraire on souhaite faire tourner un grand nombre d'agent, on est
rapidemment  limit� par  les  ressources  de la  machine  � partir  de
quelques milliers d'agents.

Toutefois, cette  flexibilit� reste avant tout un  avantage malgr� ces
limites.


%%% Local Variables: 
%%% mode: latex 
%%% TeX-master:  "rapport" 
%%% TeX-PDF-mode: t 
%%% coding:latin-1 
%%% End:
