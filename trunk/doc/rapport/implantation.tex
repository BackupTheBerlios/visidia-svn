Dans cette  partie, nous  allons nous attacher  � d�velopper  de facon
d�taill�e l'implantation de notre  module au sein du logiciel \visidia
existant.\\

Le projet s'est subdivis� en deux parties essentielles :
\begin{itemize}
\item le d�veloppement de l'API et de la simulation.
\item  l'int�gration  des  nouvelles  fontionnalit�s  dans  la  partie
graphique de \visidia.
\end{itemize}


\section{Agents}
Cette section  concerne le d�veloppement  des agents et donc  de l'API
fournie aux clients. Le description  de cette API ayant d�j� �t� faite
dans la  partie pr�c�dente  nous ne nous  attarderons pas  dessus ici.
Elle est  enti�rement d�finie et  implant�e dans la  classe Agent.java
qui se trouve dans  le package visidia.simulation.agents. 

Les m�thodes de cette classe sont bas�es sur des appels au m�thodes du
simulateur qui correspondent.  En effet, comme c'est le simulateur qui
fait le lien  entre l'interface graphique et les agents  il est tout a
fait normal qu'il ait la charge des actions effectu�es par les agents.

En plus  du simlateur  duquel il  d�pend et de  son idetit�,  un agent
contient une  struture de  donn�e de type  WhiteBoard d�finie  dans le
package visidia.tools.agents  et qui permet  � l'agent de  stocker des
informations durant son ex�cution.  Un WhiteBoard s'utilise de la m�me
mani�re qu'une table de  hachage, la diff�rence �tant qu'un WhiteBoard
permet un acc�s a des valeurs par d�faut
\section{Simulateur}


\section{GUI}





%% Local Variables: 
%% mode: latex
%% TeX-master: t
%% TeX-PDF-mode: t
%% coding: latin-1
%% End: