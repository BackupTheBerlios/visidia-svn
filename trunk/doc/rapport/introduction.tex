L'algorithmique distribu�e devient de nos jours de plus en plus
importante aux vues des utilisations possibles sur les r�seaux et
autres syst�mes distribu�s. 
A chaque fois que l'on souhaite augmenter la capacit�  de calcul (� moindre
coup) en ajoutant graduellement des entit�s, ou regrouper des ensembles 
d'entit�s d�j� existant; ou bien; que l'on souhaite faire
communiquer entre elles des entit�s distantes (Telecommunication, R�seau LAN,
Satellite, Wifi, ...) et mettre en rapport des 
banques de donn�es r�parties (Internet, Peer To Peer), nous avons recours a
des algorithmes distribu�s.\\

Cependant, ces applications distribu�es sont difficiles � implanter, tester 
et exp�rimenter.  En effet, celles-ci,
en  plus  des  probl�mes  des  applications  classiques  centralis�es,
doivent  g�rer  des probl�mes  de communication  entre processus,  de
concurrence  et de  conflits  de  ressources.  C'est  dans  le but  de
faciliter la t�che � un d�veloppeur d'algorithmes  distribu�s que des
chercheurs  du  LaBRI\footnote{\url{http://www.labri.fr}}  ont   cr��
\visidia. 

Ce logiciel est un atelier proposant des outils
de simulation  et  de visualisation au travers  d'animations 
graphiques en temps r�el et proposant l'ex�cution d'un algorithme distribu� sur
un graphe  donn�.  Cet algorithme peut �tre  choisi dans une liste,
algorithme d�velopp� par  l'utilisateur, ou encore  dessin� gr�ce � des
r�gles de r��criture.

L'application  va  donc  constituer  �  terme un  outil permettant la 
validation pratique de r�sultats th�oriques sur un environnement
distribu�.

L'objet de notre projet est d'ajouter de nouvelles fonctionnalit�s �
l'application existante et d'impl�menter des algorithmes distribu�es pour
enrichir la collection d'algorithmes fonctionnannt sous \visidia.

