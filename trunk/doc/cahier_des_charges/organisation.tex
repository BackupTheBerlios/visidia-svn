\chapter{Organisation}

\section{Planning des taches}

%%   - Diagramme de Gantt si possible
%%   ;;planning des rendez-vous
%%   - planning des �tapes de remise du projet

L'�quipe  des r�alisateurs rencontrera  les clients  une fois  par semaine
pour faire le point sur le travail effectu�. L'horaire du vendredi matin �
10h  a �t� retenu  pour cette  entrevue hebdomadaire.  Les clients  et les
r�alisateurs resterons  en contact et  pourrons �changer sur le  projet au
moyen d'une liste de diffusion \ref{mailing-list}.

\section{�ch�ancier}

La date de remise des rapports de fin de PFA aux clients est fix�e au
vendredi 14 avril, 17h00 derni�re limite.

La remise du projet donnera lieu � une soutenance dont la date est
fix�e au jeudi 27 avril.

\section{Description des outils}
%% - D�ploiement (cd,dvd,autre)
%% - Presentation du portail, du site, de la mailing liste, etc.
%% - methode de validation du travail (tests, iteratif ou unitaire)

\subsection{Utilisation d'une plate-forme de d�veloppement}

Le projet sera l'occasion d'utiliser un portail �volu� de
d�veloppement sur le mod�le du c�l�bre portail
%\href{http://sourceforge.net}{SourceForge}.
SourceForge \cite{SourceForge}.
Notre choix s'est port� sur le portail
%\href{http://www.berlios.de}{\berlios} 
\berlios \cite{BerliOs}  pour des raisons de rapidit�  des serveurs. Notre
projet \visidia dispose ainsi d'une page d'accueil \cite{visidiaPFA}:
%\href{http://visidia.berlios.de}{\visidia.\berlios} 
\begin{verbatim}
http://visidia.berlios.de
\end{verbatim}

Cette page  d'accueil est un  wiki \cite{wiki}. Cela permet  une �volution
plus souple des informations pr�sent�es. Cette page est le point de d�part
vers  les  diff�rents services  fournis  par  \berlios.  Tous les  modules
d�crits par la suite peuvent �tre  retrouv�s � partir de la page d'accueil
du projet.\\

Tout d'abord les sources de notre extension de \visidia sont
maintenues gr�ce au gestionnaire de version Subversion
\cite{subversion}. Le client a la possibilit� de suivre l'avanc�e du
projet en acc�dant aux sources. Deux interfaces web permettent l'acc�s
aux sources: 
%\href{http://svn.berlios.de/viewcvs/visidia/}{viewcvs}
viewcvs \cite{viewcvs} 
et
%\href{http://svn.berlios.de/wsvn/visidia/}{websvn}
websvn \cite{websvn}. Il est �galement possible de t�l�charger anonymement
les sources depuis le d�p�t Subversion:
\begin{verbatim}
svn checkout svn://svn.berlios.de/visidia/trunk
\end{verbatim}

L'�quipe  a aussi mis  en place  des mailing-lists  g�r�es par  le portail
\berlios.   Une  mailing-list  permet  aux  clients de  joindre  tous  les
r�alisateurs  et vice  versa. L'adresse  de cette  liste de  diffusion est
\label{mailing-list}:
\begin{verbatim}
visidia-pfa@berlios.de
\end{verbatim}

Il est possible  de consulter les archives de cette  liste de diffusion au
moyen d'une interface web \cite{mailing-list}. 
 

%% Local Variables:
%% mode: latex
%% coding: latin-1
%% TeX-master: "main"
%% End: