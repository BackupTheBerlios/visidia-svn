%%%%%%NOM de UC : Lancer simulation
\section{Lancer simulation}
\subsection{Cartouche d'identification de la fiche de cas d'utilisation}

\begin{table}[h]
\begin{center}
\begin{tabular}{|l|l|}
\hline
Nom fiche de cas d'utilisation & Fiche\_CULancerSimulation \\ \hline
Date de mise a jour &  25/Jan/2006\\ \hline
Version & 1.0 \\ \hline
Redacteur & GHRISS.PFA-VISIDIA\\ \hline
Priorit� du cas d'utilisation & Haute\\
\hline
\end{tabular}
\end{center}
\caption{} \label{cartouche_usecase}
\end{table}


\subsection{Description du cas d'utilisation}

\subsubsection{G�n�ralit�s}

\begin{table}[ht!]
\begin{center}
\begin{tabular}{|l|l|}
\hline
Cas d'utilisation & Lancer Simulation\\ \hline
Acteurs &  Le client \\ \hline
But & Permettre � l'utilisateur de lancer la visualisation de l'execution de l'algorithme \\ \hline
R�sum� M�tier & L'utilisateur clique sur un bouton qui va declancher le debut de la simulaion\\ \hline
Pr� conditions & L'utilisateur a dessin� son graphe, il a choisi �galement son algorithme.\\
& L'utilisateur a choisi un noeud pour lui affecter un agent.\\ \hline
Post conditions & La simulation de l'execution de l'algorithme sur le graphe est d�clanch�.\\ \hline
Commentaires & \\ 
\hline
\end{tabular}
\end{center}
\caption{} \label{genralites_usecase}
\end{table}

\subsubsection{Sc�nario Nominal}

\begin{table}[ht!]
\begin{center}
\begin{tabular}{|c|l|}
\hline
Num�ro enchainement & Action \\  \hline
1 & L'utlisateur desine son graphe\\ \hline
2 & L'utlisateur choisi son algorithme(Son Agent)\\ \hline
3 & L'utlisateur parametre le graphe (affectation des agents) \\ \hline
4 & L'utlisateur clique sur le bouton lan�ant la simulation\\ \hline
5 & Le systeme commence l'execution de l'algorithme (par quoi ?)\\
\hline
\end{tabular}
\end{center}
\caption{} \label{nominal_usecase}
\end{table}

\subsubsection{Sc�nario Alternatif}

\begin{table}[ht!]
\begin{center}
\begin{tabular}{|l|l|}
\hline
N enchainement & Action \\ 
\hline
Point 3 :3.1 & Le Graphe n'est pas param�tr�e.\\ \hline
3.1.1 & Le systeme demande � l'utilisateur de parametrer le graphe.\\ \hline
3.1.2 & L'utlisateur param�tre son graphe. \\ \hline
3.1.3 & L'utlisateur relance la simulation. \\ \hline
3.1.4 & La simulation est d�clanch�e.\\
 \hline
\end{tabular}
\end{center}
\caption{} \label{alternatif_usecase}
\end{table}

\subsubsection{Exceptions}

\begin{table}[ht!]
\begin{center}
\begin{tabular}{|l|l|}
\hline
N enchainement & Action \\ 
\hline
1& La simulation est d�j� lacnc�e.\\
 \hline
\end{tabular}
\end{center}
\caption{} \label{exceptions_usecase}
\end{table}

