%%%%%%NOM de UC : Lancer simulation
\section{Lancer simulation}
%% \subsection{Cartouche d'identification de la fiche de cas d'utilisation}

%% \begin{table}[h]
%% \begin{tabular}{|l|l|}
%% \hline
%% Nom fiche de cas d'utilisation & Fiche\_CULancerSimulation \\ \hline
%% Date de mise a jour &  25/Jan/2006\\ \hline
%% Version & 1.0 \\ \hline
%% Redacteur & GHRISS.PFA-VISIDIA\\ \hline
%% Priorit� du cas d'utilisation & Haute\\
%% \hline
%% \end{tabular}
%% \caption{} \label{cartouche_usecase}
%% \end{table}

\subsection{Description du cas d'utilisation}

\subsubsection{G�n�ralit�s}

\begin{table}[h!t]
\begin{tabular}{|l|l|}
\hline
Cas d'utilisation & Lancer Simulation\\ \hline
Acteurs &  Le client \\ \hline
But & Permettre � l'utilisateur de lancer la visualisation de l'ex�cution de l'algorithme \\ \hline
R�sum� M�tier & L'utilisateur clique sur un bouton qui va d�clencher le d�but de la simulation\\ \hline
Pr� conditions & L'utilisateur a dessin� son graphe, il a choisi �galement son algorithme.\\
& L'utilisateur a choisi un noeud pour lui affecter un agent.\\ \hline
Post conditions & La simulation de l'ex�cution de l'algorithme sur le graphe est d�clench�.\\ \hline
Commentaires & \\ 
\hline
\end{tabular}
%\caption{} 
\label{genralites_lancerSimulation}
\end{table}

\subsubsection{Sc�nario Nominal}

\begin{table}[h!t]
\begin{tabular}{|c|l|}
\hline
Num�ro encha�nement & Action \\  \hline
1 & L'utilisateur dessine son graphe\\ \hline
2 & L'utilisateur choisi son algorithme(Son Agent)\\ \hline
3 & L'utilisateur param�tre le graphe (affectation des agents) \\ \hline
4 & L'utilisateur clique sur le bouton lan�ant la simulation\\ \hline
5 & Le syst�me commence l'ex�cution de l'algorithme (par quoi ?)\\
\hline
\end{tabular}
%\caption{}
\label{nominal_lancerSimulation}
\end{table}

\subsubsection{Sc�nario Alternatif}

\begin{table}[h!t]
\begin{tabular}{|c|l|}
\hline
N encha�nement & Action \\ 
\hline
Point 3 :3.1 & Le Graphe n'est pas param�tr�e.\\ \hline
3.1.1 & Le syst�me demande � l'utilisateur de param�trer le graphe.\\ \hline
3.1.2 & L'utilisateur param�tre son graphe. \\ \hline
3.1.3 & L'utilisateur relance la simulation. \\ \hline
3.1.4 & La simulation est d�clench�e.\\ \hline
\end{tabular}
%\caption{} 
\label{alternatif_lancerSimulation}
\end{table}

\subsubsection{Exceptions}

\begin{table}[h!t]
\begin{tabular}{|c|l|}
\hline
N encha�nement & Action \\ 
\hline
1& La simulation est d�j� lanc�e.\\ \hline
\end{tabular}
%\caption{} 
\label{exceptions_lancerSimulation}
\end{table}

