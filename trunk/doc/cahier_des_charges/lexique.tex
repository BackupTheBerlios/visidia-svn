\chapter{Lexique}

Vous  trouverez  ci-dessous les  d�finitions  relatives aux  principaux
termes et sigles utilis�s dans la r�alisation du cahier des charges.


\begin{description}

\item[Algorithmique distribu�e] : Cette partie de l'algorithmique
s'int�resse aux algorithmes s'ex�cutant sur plusieurs unit�s de calcul
en m�me temps. Le mod�le th�orique d'un r�seau dans ce cadre est un
graphe dont les sommets (ou noeuds) sont des machines ou processeurs
et les ar�tes des connexions.

\item[Agent   mobile]   :   Entit�   autonome   de   calcul   qui   se
d�place \cite{agentBook}.

\item[A.P.I.]   :  Application  Programming  Interface.   Ensemble  de
prototypes   de  fonctions   accessibles   depuis  l'ext�rieur   d'une
biblioth�que. C'est l'interface publique de la biblioth�que.

\item[Door  ou porte]  : une  porte est  une connexion  �  partir d'un
sommet du graphe  vers un autre sommet.  Les  portes sont num�rot�es �
partir de 0.

\item[ENSEIRB]   :    �cole   Nationale   Sup�rieure   d'�lectronique,
Informatique et Radiocommunications de Bordeaux \cite{enseirb}.

\item[Javadoc] :  outil qui  permet de g�n�rer  de la  documentation �
  partir du code source et des commentaires Java.

\item[LaBRI]  :  Laboratoire Bordelais  de  Recherche en  Informatique
\cite{labri}.

\item[PFA]  :  Projet  de  Fin  d'Ann�e. Module  de  programmation  de
  quatri�me semestre  de fili�re informatique �  l'ENSEIRB.  Ce module
  consiste en  l'�laboration d'un  projet sur une  dur�e de 3  mois en
  groupe de 7  � 8 personnes. Il a pour but  de nous familiariser avec
  les m�thodes  de travail  en groupe sur  des projets de  plus longue
  dur�e  et   de  plus  grande   taille  que  ceux  dont   nous  avons
  l'habitude. Dans le cadre du pr�sent projet, les chercheurs du LaBRI
  constituent donc les clients auxquels  on doit rendre des comptes en
  tant que prestataires.

\item[Thread  ou processus  l�ger]  :  un thread  est  similaire �  un
  processus : il ex�cute  un ensemble d'instructions. Toutefois, l� ou
  chaque processus  poss�de sa  propre m�moire virtuelle,  les threads
  qui appartiennent au m�me processus p�re partagent un m�me partie de
  sa m�moire virtuelle.

\item[\visidia]   :  Visualization   and  Simulation   of  Distributed
Algorithms ou  Visualisation et Simulation  des Algorithmes Distribu�s
\cite{visidiaLaBRI}.

\end{description}



%% Local Variables:
%% mode: latex
%% coding: latin-1
%% TeX-master: "main"
%% End: