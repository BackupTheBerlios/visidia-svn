\chapter{Lexique}

Vous  trouverez  ci-dessous les  d�finitions  relatives aux  princpaux
termes et sigles utilis�s dans la r�alisation du cahier des charges.


\begin{description}

\item[A.P.I.]   :  Application  Programming  Interface.   Ensemble  de
prototypes   de  fonctions   accessibles   depuis  l'ext�rieur   d'une
biblioth�que. C'est l'interface publique de la biblioth�que.

\item[Door  ou porte]  : une  porte est  une connexion  �  partir d'un
sommet du graphe  vers un autre sommet.  Les  portes sont num�rot�es �
partir de 0.

\item[ENSEIRB]   :    Ecole   Nationale   Sup�rieure   d'El�ctronique,
Informatique et Radicommunications de Bordeaux \cite{enseirb}.

\item[LaBRI]  :  Laboratoire Bordelais  de  Recherhce en  Informatique
\cite{labri}.

\item[\visidia]   :  Visualization   and  Simulation   of  Distributed
Algorithms ou  Visualisation et Simulation  des Algorithmes Distribu�s
\cite{visidiaLaBRI}.

\end{description}



%% Local Variables:
%% mode: latex
%% coding: latin-1
%% TeX-master: "main"
%% End: