\section{Param�trer Simulateur}
%\subsection{Cartouche d'identification de la fiche de cas d'utilisation}

%\begin{table}[ht!]
%\begin{tabular}{|l|l|}
%\hline
%Nom fiche de cas d'utilisation & DefineRewritingRule \\  \hline
%Date de mise a jour & $LastChangedDate$ \\ \hline
%Version & 0.1 \\ \hline
%Redacteur & Xavier\\ \hline
%Priorit� du cas d'utilisation & Moyen \\ \hline
%\end{tabular}
%\caption{} \label{cartouche_ReglesReecriture}
%\end{table}

\subsection{Description du cas d'utilisation}

\subsubsection{G�n�ralit�s}

\begin{table}[ht!]
\begin{tabular}{|l|l|}
\hline
Cas d'utilisation & Param�trer le simulateur. \\ \hline
Acteurs & L'utilisateur. \\ \hline
But & Permettre � l'utilisateur de d�finir le comportement du simulateur. \\ \hline
R�sum� M�tier & L'utilisateur d�fini ses r�gles de r��criture de
mani�re ``graphique''. \\ \hline
Pr� conditions & La partie simulation est lanc�e. \\ \hline
Post conditions & Le simulateur est param�trer. \\ \hline
Commentaires & Le param�trage s'applique seulement au simulateur
courant du syst�me. Possibilit� de pouvoir stocker la configuration ??  \\ \hline
\end{tabular}
%\caption{} 
\label{genralites_ParamSimu}
\end{table}

\subsubsection{Sc�nario Nominal}

\begin{table}[ht!]
\begin{tabular}{|c|l|}
\hline
Num�ro enchainement & Action \\  \hline
1 & L'utilisateur saisie les param�tres du simulateur : 
\begin{itemize}
\item Les sommets ont un identifiant : Oui/Non
\item Les agents ont un identifiant : Oui/Non
\item Choix de la r�partition : Choix � la souris (default) /
  Al�atoire : Proba / Sp�cifique : choix de la classe.
\end{itemize}
\\ \hline
2 & Le syst�me enregistre les param�tres du simulateur\\ \hline
\hline
\end{tabular}
%\caption{}
\label{nominal_ReglesReecriture}
\end{table}

\subsubsection{Exceptions}

Pas de sc�nario d'exception.
