\section{Param�trer Simulateur}
%\subsection{Cartouche d'identification de la fiche de cas d'utilisation}

%\begin{table}[ht!]
%\begin{tabular}{|l|l|}
%\hline
%Nom fiche de cas d'utilisation & DefineRewritingRule \\  \hline
%Date de mise a jour & $LastChangedDate$ \\ \hline
%Version & 0.1 \\ \hline
%Redacteur & Xavier\\ \hline
%Priorit� du cas d'utilisation & Moyen \\ \hline
%\end{tabular}
%\caption{} \label{cartouche_ReglesReecriture}
%\end{table}

\subsection{Description du cas d'utilisation}

\subsubsection{G�n�ralit�s}

\begin{table}[h!t]
\begin{tabular}{|p{3cm}|p{10cm}|}
\hline
Cas d'utilisation & Param�trer le simulateur. \\ \hline
Acteurs & L'utilisateur. \\ \hline
But & Permettre � l'utilisateur de d�finir le comportement du simulateur. \\ \hline
R�sum� M�tier & L'utilisateur d�fini ses r�gles de r��criture de
mani�re ``graphique''. \\ \hline
Pr� conditions & La partie simulation est lanc�e. \\ \hline
Post conditions & Le simulateur est param�trer. \\ \hline
\multirow{2}{0cm}{Commentaires} & Le param�trage s'applique seulement au simulateur
courant du syst�me.\\ \hline
\end{tabular}
%\caption{} 
\label{genralites_ParamSimu}
\end{table}

\subsubsection{Sc�nario Nominal}

\begin{table}[h!t]
\begin{tabular}{|c|p{10cm}|}
  \hline
  Num�ro encha�nement & Action \\  \hline
  \multirow{5}{0cm}{1} & L'utilisateur saisie les param�tres du
  simulateur : \\
  & Les sommets ont un identifiant : Oui/Non \\
  & Les agents ont un identifiant : Oui/Non \\
  & Choix de la r�partition : Choix � la souris (d�faut) /
   Al�atoire : Proba / Sp�cifique : choix de la classe. \\ \hline
  2 & Le syst�me enregistre les param�tres du simulateur \\ \hline
\end{tabular}
%\caption{}
\label{nominal_ParamSimu}
\end{table}

\subsubsection{Exceptions}

Pas de sc�nario d'exception.
