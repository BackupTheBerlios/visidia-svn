\subsection{Choisir Graphe}

Ce cas d'utilisation va permettre � l'utilisateur de charger un graphe
dans le syst�me. Ce cas d'utilisation sera accessible
seulement dans la fenetre d'�dition du graphe.


%\subsection{Cartouche d'identification de la fiche de cas d'utilisation}

%\begin{table}[ht!]
%\begin{tabular}{|l|l|}
%\hline
%Nom fiche de cas d'utilisation & OpenGraph \\  \hline
%Date de mise a jour & $LastChangedDate$ \\ \hline
%Version & 0.1 \\ \hline
%Redacteur & Xavier\\ \hline
%Priorit� du cas d'utilisation & Faible \\
%\hline
%\end{tabular}
%\caption{} \label{cartouche_choisirGraphe}
%\end{table}


\subsubsection{G�n�ralit�s}

%\begin{table}[h!t]
\begin{tabular}{|l|l|}
\hline
Cas d'utilisation & Choisir un graphe. \\ \hline
Acteurs & L'utilisateur. \\ \hline
But & Permettre � l'utilisateur de sp�cifier le graphe sur lequel on
va travailler. \\ \hline
R�sum� M�tier & L'utilisateur indique le fichier graphe (format GML). \\ \hline
Pr� conditions & L'utilsateur dispose de graphes. \\ \hline
Post conditions & Le graphe est charg�. \\ \hline
Commentaires & Un seul graphe ne peut �tre charger � la fois.\\ \hline
\end{tabular}
%\caption{}
%\label{genralites_choisirGraphe}
%\end{table}

\subsubsection{Sc�nario Nominal}

%\begin{table}[h!t]
\begin{tabular}{|c|l|}
\hline
Num�ro enchainement & Action \\  \hline
1 & Le syst�me pr�sente le choix des graphes \\ \hline
2 & L'utilisateur s�lectionne le graphe \\ \hline
3 & Le syst�me charge le graphe\\ \hline
\end{tabular}
%\caption{} 
%\label{nominal_choisirGraphe}
%\end{table}

\subsubsection{Sc�nario Alternatif}

Pas de sc�nario alternatif

\subsubsection{Exceptions}

%\begin{table}[h!t]
\begin{tabular}{|c|l|}
\hline
N enchainement & Action \\ \hline
Au point 2: 2.1 & L'utilisateur annule\\ \hline
 2.1.1 & Fin du cas d'utilisation\\ \hline
\end{tabular}
%\caption{}
%\label{exceptions_choisirGraphe}
%\end{table}
