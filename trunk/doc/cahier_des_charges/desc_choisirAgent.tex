\section{Choisir Agent}
%\subsection{Cartouche d'identification de la fiche de cas d'utilisation}

%\begin{table}[h!]
%\begin{tabular}{|l|l|}
%\hline
%Nom fiche de cas d'utilisation & OpenAgent \\  \hline
%Date de mise a jour & $LastChangedDate$ \\ \hline
%Version & 0.1 \\ \hline
%Redacteur & Xavier\\ \hline
%Priorit� du cas d'utilisation & Faible \\
%\hline
%\end{tabular}
%\caption{} \label{cartouche_choisirAgent}
%\end{table}


\subsection{Description du cas d'utilisation}

\subsubsection{G�n�ralit�s}

\begin{table}[h!]
\begin{tabular}{|l|l|}
\hline
Cas d'utilisation & Choisir un Agent. \\ \hline
Acteurs & L'utilisateur. \\ \hline
But & Permettre � l'utilisateur de sp�cifier les agents qui vont
op�rer sur le graphe. \\ \hline
R�sum� M�tier & L'utilisateur indique le fichier agent (format
byteCode Java). \\ \hline
Pr� conditions & L'utilsateur dispose d'impl�mentation d'agents. \\ \hline
Post conditions & L'agent est charg�. \\ \hline
Commentaires &  Un seule impl�mentation d'agent peut �tre
s�lectionn�. !! A VOIR !!\\ \hline
\end{tabular}
%\caption{} 
\label{genralites_choisirAgent}
\end{table}

\subsubsection{Sc�nario Nominal}

\begin{table}[h!]
\begin{tabular}{|c|l|}
\hline
Num�ro enchainement & Action \\  \hline
1 & Le syst�me pr�sente le choix des agents \\ \hline
2 & L'utilisateur s�lectionne l'agent \\ \hline
3 & Le syst�me charge l'agent\\ \hline
\hline
\end{tabular}
%\caption{}
\label{nominal_usecase}
\end{table}

\subsubsection{Sc�nario Alternatif}

Pas de sc�nario alternatif

\subsubsection{Exceptions}

\begin{table}[h!]
\begin{tabular}{|l|l|}
\hline
N enchainement & Action \\ \hline
Au point 2: 2.1 & L'utilisateur annule\\ \hline
 2.1.1 & Fin du use case\\ \hline
\end{tabular}
\caption{} \label{exceptions_choisirAgent}
\end{table}
