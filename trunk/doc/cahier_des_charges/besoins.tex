\chapter{Besoins fonctionnels et non fonctionnels}

\section{Description de l'A.P.I}

Les m�thodes de la classe \emph{Agent} fournies � l'utilisateur pour
l'�criture de l'algorithme sont les suivantes :\\

\begin{itemize}
\item \emph{getArity} retourne le nombre de portes sortantes du sommet
  courant (la premi�re est num�rot�e 0)\\
\item \emph{curVertex} retourne le num�ro du sommet courant (suppose
  que l'algorithme utilise un identifiant unique pour chaque sommet)\\
\item \emph{getNetSize} retourne le nombre de sommets du graphe\\
\item \emph{setMover} permet de d�finir pour l'agent un nouveau type
de d�placement dont le nom est pass� en param�tre\\
\item \emph{move} permet de d�placer l'agent sur la porte suivante ou
  sur la porte dont le num�ro est pass� en param�tre\\
\item \emph{moveBack} permet de d�placer l'agent sur la porte dont il
  vient\\
\item \emph{setDoorState} marque la porte pass�e en param�tre avec une
  marque, pass�e �galement en param�tre\\
\item \emph{entryDoor} retourne le num�ro de la porte par laquelle
  l'agent vient d'arriver\\
\item \emph{putVertexProperty} place une propri�t� sur le sommet
  courant, la valeur et la cl� sont pass�es en param�tre\\
\item \emph{getVertexProperty} retourne la valeur de la propri�t� dont
  la cl� est pass�e en param�tre\\
\item \emph{putDoorProperty} place une propri�t� sur une porte, le
  num�ro de la porte, la cl� et la valeur sont pass�s en param�tre\\
\item \emph{getDoorProperty} retourne la valeur de la propri�t� dont
  la cl� est pass�e en param�tre, sur la porte dont le num�ro est
  �galement pass� en param�tre\\
\item \emph{cloneAndSend} clone l'agent en cours (avec son tableau
  blanc) et envoie le clone sur la porte pass�e en param�tre\\
\end{itemize}

\section{schema}

Inserer ici un diagramme de sequence ou diagramme etat-transition


\section{Contraintes}

- Langage Java
- Code maintenable
- Code comprehensible by everyone in Da WoRlD
- Javadoc
- Standard de codage

*Utilisation des conventions JAVA
noms des classes commencant par une majuscule
noms des methodes commencent par une majuscule
majuscules pour separer les differents mots composant un nom 
indentation GNU style
accesseurs commencant par 'get'
modificateur commencant par 'set'

*Autres standards
nom des classes et des methodes en anglais
reprise au maximum des noms existant
utilisation de noms explicites
commentaires en anglais style Javadoc
 
- Licence Open Source GPL v2 ou superieure


%% Local Variables:
%% mode: latex
%% coding: latin-1
%% TeX-master: "main"
%% End: