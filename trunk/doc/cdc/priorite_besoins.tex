\subsection{Diagramme de GANTT}

** Diagramme de GANTT **

\subsection{Objectifs secondaires}

Les objectifs secondaires du projet sont les besoins du client qui ne sont
pas prioritaires et dont la r�alisation sort du cadre des engagements de
l'�quipe vis � vis du client.

Les t�ches suivantes seront donc r�alis�es si le projet a pu �tre avanc� plus 
rapidement que nous l'avions initialement pr�vu.


\subsubsection{Statistiques temps r�el}

Le client d�sirerait pouvoir consulter les statistiques de chaque agent, de chaque
n\oe ud, ainsi que celle de l'ex�cution de l'algorithme, et cela en temps r�el. 
Ces statistiques pourront �tre int�gr�es dans une fen�tre s�par�e du
programme.

\subsubsection{Choix d'action lors de la suppression d'une arr�te}

Le client voudrait pouvoir choisir dans la configuration d'une simulation
l'action � effectuer lors de la suppression d'une arr�te sur laquelle circule
un agent parmi les cas suivant :

\begin{itemize}
 \item tuer l'agent (cas par d�faut);
 \item retourner l'agent vers son sommet d'origine.
\end{itemize}

\subsubsection{Choix d'action de l'arriv�e d'un agent sur un sommet �teint}

Le client voudrait pouvoir choisir dans la configuration d'une simulation
l'action � effectuer lorsqu'un agent se d�place dans un sommet �teint
(suite � un crash ou � une extinction) parmi les cas suivant :

\begin{itemize}
 \item tuer l'agent (cas par d�faut);
 \item retourner l'agent vers son sommet d'origine.
\end{itemize}

\subsubsection{R�daction d'une documentation sur la conception d'un algorithme dans \visidia}

La r�daction d'un algorithme dans \visidia n'est pas une t�che facile et est
tr�s peu document�e. Le client voudrait qu'une documentation d�taillant comme r�diger
un algorithme dans \visidia soit r�alis�e.


