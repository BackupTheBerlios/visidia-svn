%% Besoins fonctionnels

\subsection{Ajouter une ar�te}
Ce cas d'utilisation permet � l'utilisateur d'ajouter une ar�te sur le
graphe pendant l'ex�cution d'un algorithme avec agents mobiles. 

Lors de l'activation de la fonctionnalit�, l'ex�cution est mise en
pause et l'utilisateur n'a alors plus qu'� s�lectionner les 2 sommets
distincts et encore non connect�s. Il lui est possible d'ajouter
d'autres ar�tes, tant qu'il n'aura pas relanc� l'ex�cution. 


\subsection{Supprimer une ar�te}
Ce cas d'utilisation permet � l'utilisateur de supprimer une ar�te sur
le graphe pendant l'ex�cution d'un algorithme avec agents mobiles. 

Lors de l'activation de la fonctionnalit�, l'ex�cution est mise en
pause et l'utilisateur n'a alors plus qu'� s�lectionner l'ar�te �
supprimer. S'il existe un agent sur l'ar�te en cours, deux cas se
pr�sentent : 
\begin{itemize}
  \item l'algorithme requiert la suppression de l'agent, il est donc tu�
  \item l'algorithme requiert le renvoi de l'agent � l'exp�diteur : �
    la reprise de l'ex�cution, l'agent retourne � son exp�diteur 
\end{itemize}

L'utilisateur a la possibilit� de supprimer d'autres ar�tes, tant
qu'il n'a pas relanc� l'ex�cution. 




