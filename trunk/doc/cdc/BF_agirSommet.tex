%% Besoins fonctionnels (Agir sur un sommet)

\subsection{Consulter la m�moire d'un sommet}
Ce cas d'utilisation permet � l'utilisateur de consulter la m�moire d'un sommet du graphe pendant l'ex�cution d'un algorithme avec agents mobiles.
\\
 Contrairement aux cas qui suivent ci-dessous, la consultation doit se faire se faire sans mettre l'ex�cution en pause.
\\
\\

\subsection{Modifier la memoire d'un sommet}
Ce cas d'utilisation permet � l'utilisateur de modifier la m�moire d'un sommet du graphe pendant l'ex�cution d'un algorithme avec agents mobiles.
\\
 L'ex�cution de l'algorithme est mise en pause et l'utilisateur n'a alors plus qu'� s�lectionner le sommet dont on veut modifier la m�moire. Il lui est possible de consulter d'autres sommets, tant qu'il n'aura pas relanc� l'ex�cution.
\\
\\

\subsection{Eteindre un sommet}
Ce cas d'utilisation permet � l'utilisateur d'�teindre un sommet du graphe pendant l'ex�cution d'un algorithme avec agents mobiles.
\\
 L'ex�cution est mise en pause et l'utilisateur n'a alors plus qu'� s�lectionner le sommet qu'il veut �teindre. Il lui est possible d'�teindre d'autres sommets, tant qu'il n'aura pas relanc� l'ex�cution.
\\
\\

\subsection{Allumer un sommet}
Ce cas d'utilisation permet � l'utilisateur d'allumer un sommet du graphe pendant l'ex�cution d'un algorithme avec agents mobiles. Bien entendu, un sommet ne peut �tre allume que si il �tait d�j� �teint.
\\
 L'ex�cution est mise en pause et l'utilisateur n'a alors plus qu'� s�lectionner le sommet �teint qu'il veut allumer. Il lui est possible d'allumer d'autres sommets �teints, tant qu'il n'aura pas relanc� l'ex�cution.
\\
\\
