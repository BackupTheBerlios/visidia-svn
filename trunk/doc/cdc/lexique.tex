\chapter{Lexique}

Vous  trouverez  ci-dessous les  d�finitions  relatives aux  principaux
termes et sigles utilis�s dans la r�alisation du cahier des charges.


\begin{description}

\item[Algorithmique distribu�e] : cette partie de l'algorithmique
s'int�resse aux algorithmes s'ex�cutant sur plusieurs unit�s de calcul
en m�me temps. Le mod�le th�orique d'un r�seau dans ce cadre est un
graphe dont les sommets (ou noeuds) sont des machines ou processeurs
et les ar�tes des connexions.

\item[Agent   mobile]   :   entit�   autonome   de   calcul   qui   se
d�place \cite{agentBook}.

\item[A.P.I.]   :  Application  Programming  Interface.   Ensemble  de
prototypes   de  fonctions   accessibles   depuis  l'ext�rieur   d'une
biblioth�que. C'est l'interface publique de la biblioth�que.

\item[GUI]  :  Graphical  User  Interface. Interface  qui  permet  une
  utilisation  simple  d'un  programme  : elle  int�gre  notamment  un
  environnement graphique et permet l'utilisation de la souris.

\item[Javadoc] :  outil qui  permet de g�n�rer  de la  documentation �
  partir du code source et des commentaires Java.

\item[R�gle  de  r��criture] :  r�gle  de  transformation locale  d'un
  graphe.  L'application d'une telle  r�gle entra�ne le changement, en
  fonction de  voisins, des  �tats des sommets  et des ar�tes  dont la
  configuration correspond aux conditions d'application de la r�gle.

\item[Thread  ou processus  l�ger]  :  un thread  est  similaire �  un
  processus : il ex�cute  un ensemble d'instructions. Toutefois, l� ou
  chaque processus  poss�de sa  propre m�moire virtuelle,  les threads
  qui appartiennent au m�me processus p�re partagent un m�me partie de
  sa m�moire virtuelle.

\item[\visidia]   :  Visualization   and  Simulation   of  Distributed
Algorithms  ou Visualisation  et  Simulation d'Algorithmes  Distribu�s
\cite{visidiaLaBRI}.

\end{description}



%% Local Variables:
%% mode: latex
%% coding: latin-1
%% TeX-master: "main"
%% End:
