\chapter{Lexique}

Vous  trouverez  ci-dessous les  d�finitions  relatives aux  principaux
termes et sigles utilis�s dans la r�alisation du cahier des charges.


\begin{description}

\item[Algorithmique distribu�e] : cette partie de l'algorithmique
s'int�resse aux algorithmes s'ex�cutant sur plusieurs unit�s de calcul
en m�me temps. Le mod�le th�orique d'un r�seau dans ce cadre est un
graphe dont les sommets (ou noeuds) sont des machines ou processeurs
et les ar�tes des liens de communication.

\item[Agent   mobile]   :   entit�   autonome   de   calcul   qui   se
d�place.

\item[A.P.I.]   :  \texttt{Application  Programming  Interface}.
  Ensemble  de prototypes   de  fonctions   accessibles   depuis
  l'ext�rieur   d'une biblioth�que. C'est l'interface publique
  de la biblioth�que. 

\item[GUI ou IHM]  :  \texttt{Graphical  User  Interface} ou
  \texttt{Interface Homme Machine}. Interface  qui  permet  une 
  utilisation  simple  d'un  programme  : elle  int�gre  notamment  un
  environnement graphique et permet l'utilisation de la souris.

\item[Javadoc] :  outil qui  permet de g�n�rer  de la  documentation �
  partir du code source et des commentaires Java.

\item[R�gle  de  r��criture] :  r�gle  de  transformation
  locale  sur les ar�tes et les sommets d'un graphe.  L'application
  d'une telle  r�gle entra�ne le changement, en   fonction de
  voisins, des  �tats des sommets  et des ar�tes  dont la
  configuration correspond aux conditions d'application de la
  r�gle. 

\item[\visidia]   :  \texttt{Visualization   and  Simulation   of
  Distributed Algorithms}  ou Visualisation  et  Simulation
  d'Algorithmes  Distribu�s. 

\end{description}



%% Local Variables:
%% mode: latex
%% coding: latin-1
%% TeX-master: "main"
%% End:
