Une fois \visidia lanc�, la fen�tre d'�dition de graphe appara�t :

Le graphe se construit � l'aide du clic gauche de la souris. Les
graphes peuvent �galement �tre sauvegard� gr�ce aux commandes \textbf{Save}
ou \textbf{Save as...} du menu \textbf{File}.

Dans le menu Transformation :
\begin{itemize}
\item \textbf{Complete} permet de transformer le graphe courant en un graphe complet.
\item \textbf{Delete edges} permet d'effacer les ar�tes du graphe.
\item \textbf{Change edges shape} permet de modifier la nature du graphe
  (orient� ou non orient�).
\item \textbf{Change vertices shape} permet de modifier la nature des
  sommets du graphe.
\item \textbf{Vertex renumbering} permet de renum�roter les sommets du
  graphs.
\end{itemize}

Une fois le graphe cr��, il est possible de faire ex�cuter \visidia
selon trois mode :

\begin{itemize}
\item Simulation avec des messages en local : bouton \textbf{Simulation}
\item Simulation avec des messages avec r�partition des noeuds du
  graphe sur plusieurs machines : bouton \textbf{Network Simulation}
\item Simulation avec des agents mobiles en local : bouton \textbf{Agents Simulation}
\end{itemize}


%%% Local Variables: 
%%% mode: latex
%%% TeX-master: "main"
%%% coding: latin-1
%%% TeX-PDF-mode: t
%%% End: 